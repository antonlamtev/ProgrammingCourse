\documentclass[12pt,a4paper]{report}
\usepackage[utf8]{inputenc}
\usepackage[russian]{babel}
\usepackage[OT1]{fontenc}
\usepackage{amsmath}
\usepackage{amsfonts}
\usepackage{amssymb}
\usepackage{graphicx}
\usepackage{cmap}					% поиск в PDF
\usepackage{mathtext} 				% русские буквы в формулах
%\usepackage{tikz-uml}               % uml диаграммы

% Генератор текста
\usepackage{blindtext}

%------------------------------------------------------------------------------

% Подсветка синтаксиса
\usepackage{color}
\usepackage{xcolor}
\usepackage{listings}
 
 % Цвета для кода
\definecolor{string}{HTML}{B40000} % цвет строк в коде
\definecolor{comment}{HTML}{008000} % цвет комментариев в коде
\definecolor{keyword}{HTML}{1A00FF} % цвет ключевых слов в коде
\definecolor{morecomment}{HTML}{8000FF} % цвет include и других элементов в коде
\definecolor{captiontext}{HTML}{FFFFFF} % цвет текста заголовка в коде
\definecolor{captionbk}{HTML}{999999} % цвет фона заголовка в коде
\definecolor{bk}{HTML}{FFFFFF} % цвет фона в коде
\definecolor{frame}{HTML}{999999} % цвет рамки в коде
\definecolor{brackets}{HTML}{B40000} % цвет скобок в коде
 
 % Настройки отображения кода
\lstset{
language=C, % Язык кода по умолчанию
morekeywords={*,...}, % если хотите добавить ключевые слова, то добавляйте
 % Цвета
keywordstyle=\color{keyword}\ttfamily\bfseries,
stringstyle=\color{string}\ttfamily,
commentstyle=\color{comment}\ttfamily\itshape,
morecomment=[l][\color{morecomment}]{\#}, 
 % Настройки отображения     
breaklines=true, % Перенос длинных строк
basicstyle=\ttfamily\footnotesize, % Шрифт для отображения кода
backgroundcolor=\color{bk}, % Цвет фона кода
%frame=lrb,xleftmargin=\fboxsep,xrightmargin=-\fboxsep, % Рамка, подогнанная к заголовку
frame=tblr
rulecolor=\color{frame}, % Цвет рамки
tabsize=3, % Размер табуляции в пробелах
showstringspaces=false,
 % Настройка отображения номеров строк. Если не нужно, то удалите весь блок
numbers=left, % Слева отображаются номера строк
stepnumber=1, % Каждую строку нумеровать
numbersep=5pt, % Отступ от кода 
numberstyle=\small\color{black}, % Стиль написания номеров строк
 % Для отображения русского языка
extendedchars=true,
literate={Ö}{{\"O}}1
  {Ä}{{\"A}}1
  {Ü}{{\"U}}1
  {ß}{{\ss}}1
  {ü}{{\"u}}1
  {ä}{{\"a}}1
  {ö}{{\"o}}1
  {~}{{\textasciitilde}}1
  {а}{{\selectfont\char224}}1
  {б}{{\selectfont\char225}}1
  {в}{{\selectfont\char226}}1
  {г}{{\selectfont\char227}}1
  {д}{{\selectfont\char228}}1
  {е}{{\selectfont\char229}}1
  {ё}{{\"e}}1
  {ж}{{\selectfont\char230}}1
  {з}{{\selectfont\char231}}1
  {и}{{\selectfont\char232}}1
  {й}{{\selectfont\char233}}1
  {к}{{\selectfont\char234}}1
  {л}{{\selectfont\char235}}1
  {м}{{\selectfont\char236}}1
  {н}{{\selectfont\char237}}1
  {о}{{\selectfont\char238}}1
  {п}{{\selectfont\char239}}1
  {р}{{\selectfont\char240}}1
  {с}{{\selectfont\char241}}1
  {т}{{\selectfont\char242}}1
  {у}{{\selectfont\char243}}1
  {ф}{{\selectfont\char244}}1
  {х}{{\selectfont\char245}}1
  {ц}{{\selectfont\char246}}1
  {ч}{{\selectfont\char247}}1
  {ш}{{\selectfont\char248}}1
  {щ}{{\selectfont\char249}}1
  {ъ}{{\selectfont\char250}}1
  {ы}{{\selectfont\char251}}1
  {ь}{{\selectfont\char252}}1
  {э}{{\selectfont\char253}}1
  {ю}{{\selectfont\char254}}1
  {я}{{\selectfont\char255}}1
  {А}{{\selectfont\char192}}1
  {Б}{{\selectfont\char193}}1
  {В}{{\selectfont\char194}}1
  {Г}{{\selectfont\char195}}1
  {Д}{{\selectfont\char196}}1
  {Е}{{\selectfont\char197}}1
  {Ё}{{\"E}}1
  {Ж}{{\selectfont\char198}}1
  {З}{{\selectfont\char199}}1
  {И}{{\selectfont\char200}}1
  {Й}{{\selectfont\char201}}1
  {К}{{\selectfont\char202}}1
  {Л}{{\selectfont\char203}}1
  {М}{{\selectfont\char204}}1
  {Н}{{\selectfont\char205}}1
  {О}{{\selectfont\char206}}1
  {П}{{\selectfont\char207}}1
  {Р}{{\selectfont\char208}}1
  {С}{{\selectfont\char209}}1
  {Т}{{\selectfont\char210}}1
  {У}{{\selectfont\char211}}1
  {Ф}{{\selectfont\char212}}1
  {Х}{{\selectfont\char213}}1
  {Ц}{{\selectfont\char214}}1
  {Ч}{{\selectfont\char215}}1
  {Ш}{{\selectfont\char216}}1
  {Щ}{{\selectfont\char217}}1
  {Ъ}{{\selectfont\char218}}1
  {Ы}{{\selectfont\char219}}1
  {Ь}{{\selectfont\char220}}1
  {Э}{{\selectfont\char221}}1
  {Ю}{{\selectfont\char222}}1
  {Я}{{\selectfont\char223}}1
  {і}{{\selectfont\char105}}1
  {ї}{{\selectfont\char168}}1
  {є}{{\selectfont\char185}}1
  {ґ}{{\selectfont\char160}}1
  {І}{{\selectfont\char73}}1
  {Ї}{{\selectfont\char136}}1
  {Є}{{\selectfont\char153}}1
  {Ґ}{{\selectfont\char128}}1
  {\{}{{{\color{brackets}\{}}}1 % Цвет скобок {
  {\}}{{{\color{brackets}\}}}}1 % Цвет скобок }
}
 
 % Для настройки заголовка кода
\usepackage{caption}
\DeclareCaptionFont{white}{\color{сaptiontext}}
\DeclareCaptionFormat{listing}{\parbox{\linewidth}{\colorbox{сaptionbk}{\parbox{\linewidth}{#1#2#3}}\vskip-4pt}}
\captionsetup[lstlisting]{format=listing,labelfont=white,textfont=white}
\renewcommand{\lstlistingname}{Код} % Переименование Listings в нужное именование структуры


%------------------------------------------------------------------------------

\author{А.~Ю.~Ламтев}
\title{Программирование}
\begin{document}
\maketitle
\chapter{Основные конструкции языка}
%############################################################
\section{Задание 1. Размен}
\subsection{Задание}
\hspace{\parindent}
Пользователь задает сумму денег в рублях, меньшую 100 (например, 16). Определить, как выдать эту сумму монетами по 5, 2 и 1 рубль, израсходовав наименьшее количество монет (например, 3 х 5р + 0 х 2р + 1 х 1р).
\subsection{Теоретические сведения}

%Конструкции языка, библиотечные функции, инструменты использованные при разработке приложения.
\hspace{\parindent}
При разработке приложения были задействованы следующие конструкции языка: оператор \textbf{switch}, структуры данных \textit{\textbf{struct}}, макросы препроцессора -- и были использованы функции стандартной библиотеки \textit{printf}, \textit{scanf} и \textit{puts}, определённые в заголовочном файле \textit{stdio.h}; \textit{atoi}, определённая в \textit{stdlib.h}. 

%Сведения о предметной области, которые позволили реализовать алгоритм решения задачи.
\hspace{\parindent}
Я решил, что разменять сумму денег монетами номиналом 5, 2 и 1 руб. наиболее оптимально можно следующим образом. Необходимо, чтобы монет большего номинала было больше, чем монет меньшего номинала, насколько это возможно. Это послужило основой для реализации алгоритма.

\subsection{Проектирование}
%Какие функции было решено выделить, какие у этих функций контракты, как организовано взаимодействие с пользователем (чтение/запись из консоли, из файла, из параметров командной строки), форматы файлов и др.
\hspace{\parindent}
В ходе проектирования было решено выделить четыре функции, одна из которых отвечает за логику, а остальные за взаимодействие с пользователем.
\begin{enumerate}
\item \textbf{Логика}
\begin{itemize}
\item \verb-change_by_coins()-

Эта функция вычисляет результат. Она содержит один целочисленный параметр - сумму денег, которую необходимо разменять. Возвращаемое значение имеет структурный тип, который включает 3 целочисленных поля: число монеток в 5 руб, число монеток в 2 руб и число монеток в 1 руб.
\end{itemize}
\textbf{\item Взаимодействие с пользователем}
\begin{itemize}
\item \verb-exchange_output()-

Эта функция выводит в консоль результат функции \textit{change\_by\_coins()}. Она содержит один параметр структурного типа, который включает 3 целочисленных поля: число монеток в 5 руб, число монеток в 2 руб и число монеток в 1 руб. Возвращаемое значение имеет тип \textbf{\textit{void}}.
\end{itemize}
\begin{itemize}
\item \verb-exchange_parameters()-

Эта функция отвечает за взаимодействие с пользователем при чтении данных из параметров командной строки. Она содержит 2 параметра: типа \textbf{\textit{int}} - количество аргументов командной строки и типа \textbf{\textit{char**}} - массив, содержащий эти аргументы. Считывает данные из параметров командной строки. Вызывает функцию \textit{exchange\_output()}, которая в свою очередь выводит в консоль результат. Возвращает пустое значение.
\end{itemize}
\begin{itemize}
\item \verb-exchange()-

Эта функция отвечает за взаимодействие с пользователем в интерактивном режиме. Она не имеет параметров. Выводит в консоль сообщение о том, что нужно ввести число. Осуществляет контролируемый ввод данных. Вызывает функцию \textit{exchange\_output()}, которая уже и выводит в консоль результат. Возвращает пустое значение.
\end{itemize}
\end{enumerate}
\subsection{Описание тестового стенда и методики тестирования}
%Среда, компилятор, операционная система, др.

\begin{flushleft}
\textbf{Интегрированная среда разработки:} Qt Creator 3.5.0 (opensource)

\textbf{Компилятор:} GCC 4.9.1 20140922 (Red Hat 4.9.1-10)

\textbf{Операционная система:} Debian GNU/Linux 8 (jessie) 32-бита (version 3.14.1)
\end{flushleft}
%Ручное тестирование, автоматическое, статический анализ кода, динамический.

На всех стадиях разработки приложения проходило тестирование, ручное и автоматическое. Последнее осуществлялось посредством модульных тестов \textit{Qt}, основанных на библиотеке  \textit{QTestLib}.

На финальной стадии был проведён статический анализ с помощью утилиты \textit{cppcheck}% и динамический анализ с помощью утилиты \textit{valgrid}.
\subsection{Тестовый план и результаты тестирования}
%Описание по шагам хода тестирования, с указанием соответствия или несоответствия ожидаемым результатам.
\begin{enumerate}
\item \textbf{Ручные тесты}
\begin{description}
\item[I тест]
\hspace{\parindent}
\begin{flushleft}
\begin{description}
\item[Входные данные:] 11
\item[Выходные данные:] 2 0 1
\item[Результат:] Тест успешно пройден
\end{description}
\end{flushleft}
\end{description}

\begin{description}
\item[II тест]
\hspace{\parindent}
\begin{flushleft}
\begin{description}
\item[Входные данные:] 3
\item[Выходные данные:] 0 1 1
\item[Результат:] Тест успешно пройден
\end{description}
\end{flushleft}
\end{description}

\item \textbf{Модульные тесты \textit{Qt}}
\begin{description}
\item[I тест]
\hspace{\parindent}
\begin{flushleft}
\begin{description}
\item[Входные данные:] 28
\item[Выходные данные:] 5 2 1
\item[Результат:] Тест успешно пройден
\end{description}
\end{flushleft}
\end{description}

\begin{description}
\item[II тест]
\hspace{\parindent}
\begin{flushleft}
\begin{description}
\item[Входные данные:] 44
\item[Выходные данные:] 8 0 2
\item[Результат:] Тест успешно пройден
\end{description}
\end{flushleft}
\end{description}

\item \textbf{Статический анализ \textit{cppcheck}}

Утилита \textit{cppcheck} не выявила ошибок.
\end{enumerate}
\subsection{Выводы}
%Слова от чистого сердца
\hspace{\parindent}
В ходе выполнения работы я получил опыт создания многомодульного приложения с отделением логики от взаимодействия с пользователем. Укрепил навыки в создании структурных типов. А также научился тестировать программу с помощью модульных тестов и анализировать с помощью утилиты \textit{cppcheck}.
\subsection*{Листинги}
\begin{itemize}
\item[] \verb-exchange.h-
\lstinputlisting[]
{../sources/subdirproject/lib_c/exchange.h}
\item[] \verb-exchange_of_coins_process.c-
\lstinputlisting[]
{../sources/subdirproject/lib_c/exchange_of_coins_process.c}
\item[] \verb-exchange_ui.c-
\lstinputlisting[]
{../sources/subdirproject/app_c/exchange_ui.c}
\end{itemize}

%\todo[inline]{Не забыть вставить все исходники}
%##################################################################################################################################################
%
%##################################################################################################################################################

\section{Задание 2. Ферзи}

\subsection{Задание}
\hspace{\parindent}
На шахматной доске стоят три ферзя (ферзь бьет по вертикали, горизонтали и диагоналям). Найти те пары из них, которые угрожают друг другу. Координаты ферзей вводить целыми числами.
\subsection{Теоретические сведения}
%Конструкции языка, библиотечные функции, инструменты использованные при разработке приложения.
\hspace{\parindent}
При разработке приложения были задействованы следующие конструкции языка: операторы ветвления \textbf{if} и \textbf{if-else-if}, оператор \textbf{switch}, оператор цикла с постусловием \textbf{do-while}, структуры данных \textbf{\textit{struct}} и перечисления \textit{\textbf{enum}} -- и были использованы функции стандартной библиотеки \textit{printf}, \textit{scanf}, \textit{puts}, определенные в заголовочном файле \textit{stdio.h}; функции \textit{abs} и \textit{atoi}, определенные в \textit{stdlib.h}.

%Сведения о предметной области, которые позволили реализовать алгоритм решения задачи.
\hspace{\parindent}
Сведения о том, что ферзь бьет по вертикали, горизонтали или диагоналям, стали основой для реализации алгоритма. Я понял, что два ферзя бьют друг друга в двух случаях: когда они находятся на одной вертикали или горизонатали, а значит у них есть общая соответственная координата, или когда они находятся на одной диагоняли, т.е расстояние между их соответственными координатами одинаково.
\subsection{Проектирование}
%Какие функции было решено выделить, какие у этих функций контракты, как организовано взаимодействие с пользователем (чтение/запись из консоли, из файла, из параметров командной строки), форматы файлов и др.
\hspace{\parindent}
В ходе проектирования было решено выделить шесть функций, две из которых отвечают за логику, а остальные за взаимодействие с пользователем.
\begin{enumerate}
\item \textbf{Логика}
\begin{itemize}
\item \verb-check_for_beating()-

Эта функция вычисляет, бьют два ферзя друг друга или нет. Имеет два параметра (2 ферзя) структурного типа, объединяющего два целочисленных поля - две координаты ферзя. Тип возвращаемого значения -- \textit{\textbf{int}} -- 1, если два ферзя бьют друг друга, и 0 -- в противном случае.
\end{itemize}

\begin{itemize}
\item \verb-queens_result()-

Эта функция определяет, какой ферзь, кого бьет. Имеет три параметра (3 ферзя) структурного типа, объединяющего два целочисленных поля - две координаты ферзя. Далее она несколько раз вызывает функцию \textit{check\_for\_beating()} и для каждой пары ферзей вычисляет резултат. Возвращаемое значение имеет тип \textbf{\textit{int}} -- один элемент из перечисления \textit{\textbf{enum}}, название которого характеризует результат.
\end{itemize}

\textbf{\item Взаимодействие с пользователем}

\begin{itemize}
\item \verb-input_with_check()-

Эта функция осуществляет контролируемый ввод из консоли координат ферзя. Имеет два параметра типа \textit{\textbf{int}} - две координаты ферзя. Возвращает пустое значение.
\end{itemize}
\begin{itemize}
\item \verb-display_result()-

Эта функция выводит в консоль результат функции \textit{queens\_result()}. Она принимает один параметр типа \textit{\textbf{int}} -- один элемент из перечисления \textit{\textbf{enum}}, название которого характеризует результат. Возвращаемое значение имеет тип \textit{\textbf{void}}.
\end{itemize}

\begin{itemize}
\item \verb-queens_parameters()-

Эта функция отвечает за взаимодействие с пользователем при вводе данных через параметры командной строки. Она содержит 2 параметра: типа \textbf{\textit{int}} - количество аргументов командной строки и типа \textbf{\textit{char**}} - массив, содержащий эти аргументы. Считывает данные из параметров командной строки. Вызывает функцию \textit{display\_result()}, которая выводит результат в консоль. Возвращаемое значение - \textit{\textbf{void}}.
\end{itemize}

\begin{itemize}
\item \verb-queens()-

Эта функция отвечает за взаимодействие с пользователем при запуске приложения в интерактивном режиме. Она не имеет параметров. Считывает данные из консоли с помощью функции \textit{input\_with\_check()}. Затем вызывает функцию %
\textit{display\_result()}, которая выводит результат в консоль. Возвращаемое значение - \textit{\textbf{void}}.
\end{itemize}
\end{enumerate}
\subsection{Описание тестового стенда и методики тестирования}
%Среда, компилятор, операционная система, др.

\begin{flushleft}
\textbf{Интегрированная среда разработки:} Qt Creator 3.5.0 (opensource)

\textbf{Компилятор:} GCC 4.9.1 20140922 (Red Hat 4.9.1-10)

\textbf{Операционная система:} Debian GNU/Linux 8 (jessie) 32-бита (version 3.14.1)
\end{flushleft}
%Ручное тестирование, автоматическое, статический анализ кода, динамический.

На всех стадиях разработки приложения проходило тестирование, ручное и автоматическое. Последнее осуществлялось посредством модульных тестов \textit{Qt}, основанных на библиотеке  \textit{QTestLib}.

На финальной стадии был проведён статический анализ с помощью утилиты \textit{cppcheck}% и динамический анализ с помощью утилиты \textit{valgrid}.
\subsection{Тестовый план и результаты тестирования}
%Описание по шагам хода тестирования, с указанием соответствия или несоответствия ожидаемым результатам.
\begin{enumerate}
\item \textbf{Ручные тесты}
\begin{description}
\item[I тест]
\hspace{\parindent}
\begin{flushleft}
\begin{description}
\item[Входные данные:] 3 1 4 8 2 2
\item[Выходные данные:] no\_one
\item[Результат:] Тест успешно пройден
\end{description}
\end{flushleft}
\end{description}

\begin{description}
\item[II тест]
\hspace{\parindent}
\begin{flushleft}
\begin{description}
\item[Входные данные:] 4 4 8 2 7 7
\item[Выходные данные:] OneThree
\item[Результат:] Тест успешно пройден
\end{description}
\end{flushleft}
\end{description}

\item \textbf{Модульные тесты \textit{Qt}}
\begin{description}
\item[I тест]
\hspace{\parindent}
\begin{flushleft}
\begin{description}
\item[Входные данные:] 1 2 3 4 5 6
\item[Выходные данные:] everyone
\item[Результат:] Тест успешно пройден
\end{description}
\end{flushleft}
\end{description}

\begin{description}
\item[II тест]
\hspace{\parindent}
\begin{flushleft}
\begin{description}
\item[Входные данные:] 1 6 2 6 1 3
\item[Выходные данные:] OneTwo\_OneThree
\item[Результат:] Тест успешно пройден
\end{description}
\end{flushleft}
\end{description}

\item \textbf{Статический анализ \textit{cppcheck}}

Утилита \textit{cppcheck} не выявила ошибок.
\end{enumerate}

\subsection{Выводы}
В ходе выполнения работы я получил опыт создания многомодульного приложения с отделением логики от взаимодействия с пользователем. Впервые использовал перечисления \textit{\textbf{enum}}, что оказалось очень удобно. Укрепил навыки в создании структурных типов, тестировании программы с помощью модульных тестов и анализе утилитой \textit{cppcheck}.
\subsection*{Листинги}
\begin{itemize}
\item[] \verb-queens.h-
\lstinputlisting[]
{../sources/subdirproject/lib_c/queens.h}
\item[] \verb-queens_check_for_beating.c-
\lstinputlisting[]
{../sources/subdirproject/lib_c/queens_check_for_beating.c}
\item[] \verb-queens_result_for_output.c-
\lstinputlisting[]
{../sources/subdirproject/lib_c/queens_result_for_output.c}
\item[] \verb-queens_ui.c-
\lstinputlisting[]
{../sources/subdirproject/app_c/queens_ui.c}
\end{itemize}
%##################################################################################################################################################
%
%##################################################################################################################################################
\chapter{Циклы}
\section{Задание 1. Деление уголком}
\subsection{Задание}
Даны натуральные числа M и N. Вывести на экран процесс их деления с остатком
\subsection{Теоритические сведения}
%Конструкции языка, библиотечные функции, инструменты использованные при разработке приложения.

%Сведения о предметной области, которые позволили реализовать алгоритм решения задачи.
\subsection{Проектирование}
%Какие функции было решено выделить, какие у этих функций контракты, как организовано взаимодействие с пользователем (чтение/запись из консоли, из файла, из параметров командной строки), форматы файлов и др.
\subsection{Описание тестового стенда и методики тестирования}
%Среда, компилятор, операционная система, др.

%Ручное тестирование, автоматическое, статический анализ кода, динамический.
\subsection{Тестовый план и результаты тестирования}
%Описание по шагам хода тестирования, с указанием соответствия или несоответствия ожидаемым результатам.
\subsection{Выводы}


\end{document}
